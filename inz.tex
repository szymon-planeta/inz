\documentclass[eng]{mgr}
\usepackage{polski}
\usepackage[utf8]{inputenc}
%\usepackage{xcolor}
\usepackage[T1]{fontenc}

%pakiety do grafiki
\usepackage{graphicx}
\usepackage{psfrag} 

%Wspomaganie tabel
\usepackage{array}
\usepackage{tabularx}
\usepackage{hhline}
%Matematyka
\usepackage{amsmath}
\usepackage{amsfonts}
\usepackage{hyperref}
\usepackage{float}
%pakiet wypisujący na marginesie etykiety równań„ i rysunkóww zdefiniowanych przez \label{}, chcąc wygenerować finalną wersję dokumentu wystarczy usunąć poniższą linię
%\usepackage{showlabels}
%\newcommand{\R}{I\!\!R} %symbol liczb rzeczywistych,
%\newtheorem{theorem}{Twierdzenie}[section] %nowe otoczenie do składania twierdzenia
\usepackage{subcaption}
\usepackage{fancyref}
\title{Zastosowanie sztucznych sieci neuronowych do predykcji wyników meczów tenisa ziemnego}
\engtitle{Artificial neural networks for predicting results of tennis matches}
\author{Szymon Płaneta}
\supervisor{dr inż. Andrzej Rusiecki} 
\field{Automatyka i Robotyka (AIR)}
\specialisation{Technologie Informacyjne \\ w Systemach Automatyki (ART)}

\begin{document}
\maketitle
\tableofcontents

\chapter{Streszczenie}

\chapter{Wstęp}

\section{Tenis}
Tenis ziemny to olimpijski sport rakietowy rozgrywany pojedynczo (singiel), w dwuosobowych zespołach jednej płci (debel) lub obu płci (mikst). Gra w tenisa w formie jaką mamy dzisiaj wywodzi się z Birmingham w Anglii. Początki tego sportu datowane są na lata 60. XIX wieku. Najstarszy turniej na świecie, Wimbledon, pierwszy raz odbył się w roku 1877, a w latach 1896-1924 tenis był dyscypliną olimpijską. Na listę olimpiady został ponownie wpisany w roku 1988.

Dzisiaj tenis to jedna z najbardziej popularnych dyscyplin sportowych na świecie. W meczu tenisa może wziąć udział każdy, kto jest w stanie trzymać rakietę. Dzięki temu ilość graczy rekreacyjnych można liczyć w milionach. Tenis jest sportem widowiskowym, dlatego zmagania profesjonalistów dostarczają wielu wrażeń i gromadzą liczną widownię. Najbardziej prestiżowymi turniejami w sezonie są cztery turnieje wielkoszlemowe: Australian Open, French Open (znany również jako Roland Garros), Wimbledon oraz US Open. Kolejną rangą turniejów w tenisie męskim są turnieje z cyklu ATP World Tour: ATP World Tour Masters 1000 (cykl dziewięciu turniejów), ATP World Tour 500 i ATP World Tour 250. 

\begin{table}[H]
\centering
\caption{Turnieje wielkoszlemowe}
\label{my-label}
\begin{tabular}{|l|l|l|l|}
\hline
\textbf{Data}       & \textbf{Nazwa turnieju}     & \textbf{Miejsce rozgrywek} & \textbf{Nawierzchnia} \\ \hline
Styczeń - Luty      & Australian Open             & Melbourne                  & Twarda                \\ \hline
Maj - Czerwiec      & French Open  				  & Paryż                      & Ziemna                \\ \hline
Czerwiec - Lipiec   & Wimbledon                   & Londyn                     & Trawiasta             \\ \hline
Sierpień - Wrzesień & US Open                     & Nowy Jork                  & Twarda                \\ \hline
\end{tabular}
\end{table}

Ogromna liczba widzów przyciąga inwestorów, przez co dyscyplina ciągle się rozwija. Na przestrzeni lat ewoluowała zarówno sama dyscyplina, jak i wykorzystywany w niej sprzęt czy wyspecjalizowane metody treningowe. Zauważalny jest również bezpośredni wpływ rozwoju technologii, czego przykładem jest system Hawk-Eye pozwalający na rozstrzyganie kontrowersyjnych punktów dzięki analizie obrazu z kamer i sygnału z czujników.



\section{Uczenie maszynowe}
Uczenie maszynowe to dynamicznie rozwijająca się dziedzina nauki. Jest wykorzystywana w rozwiązywaniu różnego rodzaju problemów. Przykładowe zastosowania to rozpoznawanie mowy i pisma, systemy rekomendacyjne czy diagnostyka medyczna. Warto wspomnieć również o zachodzącym ostatnio gwałtownym postępie w sferze autonomicznych pojazdów. Ciągle powstają nowe, innowacyjne projekty, korzystające z rozmaitych metod uczenia maszynowego w nieznanych dotąd zastosowaniach. Rozwój tej dziedziny zdaje się nie mieć końca.

Istnieje wiele rodzajów uczenia maszynowego, każdy z nich ma swoje zalety i wady. Różne metody znajdują zastosowania w rozwiązywaniu różnych problemów. Niektóre ze znanych metod uczenia maszynowego to maszyna wektorów uczących, sieć bayesowska czy sieci neuronowe. W tej pracy wykorzystano ostatnią z nich - sieć neuronową, a konkretnie perceptron wielowarstwowy.

\section{Cel projektu}
Celem projektu było stworzenie sieci neuronowej, która przewidywać będzie zwycięzcę meczu tenisa ziemnego mężczyzn (ATP World Tour) na podstawie danych statystycznych dostępnych przed meczem. Zamierzano osiągnąć możliwie najlepsze wyniki, poprzez doświadczalne określenie wpływu architektury sieci oraz rodzaju danych wejściowych na jakość osiąganych rezultatów. Predykcje najlepiej spisującego się modelu będą publikowane na portalu społecznościowym Twitter.

Powodem wyboru tematu była chęć zastosowania uczenia maszynowego w kontekście tematyki sportowej. Powstawały już prace o podobnej tematyce, jednak takie wykorzystanie sieci neuronowych nadal nie jest jednym ze sztandarowych zastosowań. Kolejnym z motywów było zainteresowanie tematyką uczenia maszynowego i chęć poszerzenia wiedzy w tej dziedzinie.


\chapter{Sieci neuronowe}
\section{Wprowadzenie}
\section{Neuron sigmoidalny}
\section{Struktura wielowarstwowej sieci perceptronowej}
\section{Propagacja wsteczna}
\section{Algorytm największego spadku gradientu}

\chapter{Przygotowanie danych statystycznych}
Posiadanie odpowiedniej ilości należycie przygotowanych danych to warunek niezbędny do przeprowadzenia procesu uczenia sieci, a w rezultacie powodzenia całego eksperymentu. Zgromadzenie i odpowiednie przetworzenie danych jest często najbardziej skomplikowaną i najbardziej czasochłonną częścią projektu wykorzystującego sieci neuronowe. 

W trakcie przygotowywania danych można napotkać wiele problemów:
\begin{itemize}
\item Brak dostępnych danych lub ich zbyt mała liczba
\item Utrudniony dostęp do danych
\item Dane nieaktualne

\end{itemize}

\section{Baza danych}
oncourt
\section{Zastosowane technologie}
python
pyodbc
\section{Wykorzystane statystyki}

\chapter{Implementacja sieci neuronowej}
\section{Uniwersalny moduł MLP}
\section{Zastosowane technologie}
\section{Struktura programu}
\section{Interfejs programisty}

\chapter{Przeprowadzone doświadczenia}
\section{Porównanie wyników dla różnych architektur sieci}
\section{Porównanie wyników dla różnych wektorów danych wejściowych}
\section{Porównanie z innymi strategiami}

\chapter{Program publikujący przewidywania sieci neuronowej na portalu społecznościowym}
\section{Zastosowane technologie}
\section{Predykcje przedmeczowe}
\section{Predykcje w trakcie meczu}

\chapter{Podsumowanie}

\end{document}