\documentclass[eng]{mgr}
\usepackage{polski}
\usepackage[utf8]{inputenc}
%\usepackage{xcolor}
\usepackage[T1]{fontenc}

%pakiety do grafiki
\usepackage{graphicx}
\usepackage{psfrag} 

%Wspomaganie tabel
\usepackage{array}
\usepackage{tabularx}
\usepackage{hhline}
%Matematyka
\usepackage{amsmath}
\usepackage{amsfonts}
\usepackage{hyperref}
\usepackage{float}
%pakiet wypisujący na marginesie etykiety równań„ i rysunkóww zdefiniowanych przez \label{}, chcąc wygenerować finalną wersję dokumentu wystarczy usunąć poniższą linię
%\usepackage{showlabels}
%\newcommand{\R}{I\!\!R} %symbol liczb rzeczywistych,
%\newtheorem{theorem}{Twierdzenie}[section] %nowe otoczenie do składania twierdzenia
\usepackage{subcaption}
\usepackage{fancyref}
\title{Zastosowanie sztucznych sieci neuronowych do predykcji wyników meczów tenisa ziemnego}
\engtitle{Artificial neural networks for predicting results of tennis matches}
\author{Szymon Płaneta}
\supervisor{dr inż. Andrzej Rusiecki \\ Katedra Informatyki Technicznej} 
\field{Automatyka i Robotyka (AIR)}
\specialisation{Technologie Informacyjne \\ w Systemach Automatyki (ART)}


\begin{document}
\maketitle
\tableofcontents

\chapter{Streszczenie}

\chapter{Wstęp}
Uczenie maszynowe to dynamicznie rozwijająca się dziedzina nauki. Jest wykorzystywana w rozwiązywaniu różnego rodzaju problemów. Przykładowe zastosowania to rozpoznawanie mowy i pisma, systemy rekomendacyjne czy diagnostyka medyczna. Warto wspomnieć również o zachodzącym ostatnio gwałtownym postępie w sferze autonomicznych pojazdów. Ciągle powstają nowe, innowacyjne projekty, korzystające z rozmaitych metod uczenia maszynowego w nieznanych dotąd zastosowaniach. Rozwój tej dziedziny zdaje się nie mieć końca.

Istnieje wiele rodzajów uczenia maszynowego, każdy z nich ma swoje zalety i wady. Różne metody znajdują zastosowania w rozwiązywaniu różnych problemów. Niektóre ze znanych metod uczenia maszynowego to maszyna wektorów uczących, sieć bayesowska czy sieci neuronowe. W tej pracy wykorzystano ostatnią z nich - sieć neuronową, a konkretnie perceptron wielowarstwowy.

\section{Cel projektu}
Celem projektu było stworzenie sieci neuronowej, która przewidywać będzie zwycięzcę meczu tenisa ziemnego mężczyzn (ATP World Tour) na podstawie danych statystycznych dostępnych przed meczem. Zamierzano osiągnąć możliwie najlepsze wyniki, poprzez doświadczalne określenie wpływu architektury sieci oraz rodzaju danych wejściowych na jakość osiąganych rezultatów. Predykcje najlepiej spisującego się modelu będą publikowane na portalu społecznościowym Twitter.

Powodem wyboru tematu była chęć zastosowania uczenia maszynowego w kontekście tematyki sportowej. Powstawały już prace o podobnej tematyce, jednak takie wykorzystanie sieci neuronowych nadal nie jest jednym ze sztandarowych zastosowań. Kolejnym z motywów było zainteresowanie tematyką uczenia maszynowego i chęć poszerzenia wiedzy w tej dziedzinie.


\chapter{Sieci neuronowe}
\section{Wprowadzenie}
\section{Neuron sigmoidalny}
\section{Struktura wielowarstwowej sieci perceptronowej}
\section{Propagacja wsteczna}
\section{Algorytm największego spadku gradientu}

\chapter{Przygotowanie danych statystycznych}
\section{Baza danych}
\section{Zastosowane technologie}
\section{Wykorzystane statystyki}

\chapter{Implementacja sieci neuronowej}
\section{Uniwersalny moduł MLP}
\section{Zastosowane technologie}
\section{Struktura programu}
\section{Interfejs programisty}

\chapter{Przeprowadzone doświadczenia}
\section{Porównanie wyników dla różnych architektur sieci}
\section{Porównanie wyników dla różnych wektorów danych wejściowych}
\section{Porównanie z innymi strategiami}

\chapter{Program publikujący przewidywania sieci neuronowej na portalu społecznościowym}
\section{Zastosowane technologie}
\section{Predykcje przedmeczowe}
\section{Predykcje w trakcie meczu}

\chapter{Podsumowanie}

\end{document}